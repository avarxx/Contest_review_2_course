\documentclass[a4paper,12pt]{article}
\usepackage{amsmath}

\usepackage[T2A]{fontenc}
\usepackage[russian]{babel}

\begin{document}

\section*{Корректность}


\text{P входит в T только если }

\begin{gather*}
    \forall\,0 \le j < |P|\colon\;
    P[j] = '?' \;\lor\; P[j] = T[i+j].
\end{gather*}


При проверке этой позиции функция \texttt{isMatch(P,T,i)} сопоставляет каждый символ
P[j] с соответствующим T[i+j]\,,\ {либо стоит «?», либо символы равны,
ошибок не возникает.} Значит, \(i\) попадёт в ответ.


\textbf{}{Почему не не будет ложных срабатываний}
\begin{gather*}
    i + |P| \le |T|
    \quad\text{и}\quad
    \forall\,0 \le j < |P|\colon\;
    P[j] = '?' \;\lor\; P[j] = T[i+j].
\end{gather*}
Это когда программа добавляет \(i\) а это условие и определяет, что \(P\) 
входит в \(T\) на позиции \(i\). Следовательно, никаких неверных индексов в результат не попадает.

\end{document}
